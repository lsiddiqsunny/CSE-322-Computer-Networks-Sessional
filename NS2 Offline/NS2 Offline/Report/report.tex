\documentclass[hidelinks,12pt]{report}

\usepackage{color}
\usepackage{amsmath}
\usepackage{graphicx}
\usepackage{wrapfig}
\usepackage{hyperref}
\usepackage{listings}

\begin{document}
\begin{titlepage}
    \begin{center}
        \vspace*{1cm}
        
        \Large
         CSE 322 ASSIGNMENT 3\\
         Network Simulator 2  \\
         How to Configure and Work with NS2\\
        \normalsize
        \vspace{1.5cm}
        Mohammed Latif Siddiq\\
        Student ID : 1505069
        
        \vfill
        
        \vspace{0.8cm}
        
        \includegraphics[width=0.2\textwidth]{buet_logo.png}
        
        \Large
        Department of Computer Science and Engineering\\
       Bangladesh University of Engineering and Technology\\
      (BUET)\\
      Dhaka 1000\\
       \today
        
    \end{center}
\end{titlepage}

\newpage

\tableofcontents
\chapter{Install NS2}
\textbf{NS2} is an open-source simulation tool that runs on Linux. It is a discreet event simulator targeted at networking research and provides substantial support for simulation of routing, multicast protocols and IP protocols, such as \textbf{UDP}, \textbf{TCP}, \textbf{RTP} and \textbf{SRM} over wired and wireless (local and satellite) networks.
\section{Download NS2}
In the very beginning step,you have to download the \textbf{ns-allinone-2.35} packeage.
You can download it from here : \href{https://sourceforge.net/projects/nsnam/}{\textbf{NS2 Download}}

The current version is 2.35.The downloaded file name should be : \textbf{ns-allinone-2.35.tar.gz}
\section{Install Required Dependencies}
NS2 requires gcc compiler,tcl,xgraph,make software.
Try following bash command one by one:
\begin{lstlisting}[language=bash]
  $ sudo apt-get update
  $ sudo apt-get dist-upgrade
  $ sudo apt-get update
  $ sudo apt-get gcc
  $ sudo apt-get install build-essential autoconf automake
  $ sudo apt-get install tcl8.5-dev tk8.5-dev
  $ sudo apt-get install perl xgraph libx11-dev libxmu-dev
\end{lstlisting}
These command will install all necessary software.
\section{Install Main Software}
Move to file where you downlaod \textbf{ns-allinone-2.35.tar.gz}.
Then try following bash command:
\begin{lstlisting}[language=bash]
  $ tar -zxvf ns-allinone-2.35.tar.gz
  $ cd ns-allinone-2.35
  $ ./install
\end{lstlisting}
You may get this error:
\begin{lstlisting}[language=bash]
linkstate/ls.h: In instantiation of ‘void LsMap<Key, T>::eraseAll() [with Key = int; T = LsIdSeq]’:
linkstate/ls.cc:396:28:   required from here
linkstate/ls.h:137:58: error: ‘erase’ was not declared in this scope, and no declarations were found by argument-dependent lookup at the point of instantiation [-fpermissive]
  void eraseAll() { erase(baseMap::begin(), baseMap::end()); }
\end{lstlisting}
In this case,you have to edit \textbf{ls.h} file.After running above command,we should find a folder named \textbf{ns-allinone-2.35}.In this folder,we can find a folder \textbf{ns-2.35}.In that folder,there is a folder named \textbf{linkstate}.You should find a  file \textbf{ls.h} and edit line number 137.

From 
\begin{lstlisting}[language=c]
  void eraseAll() { erase(baseMap::begin(), baseMap::end()); }
\end{lstlisting}
To
\begin{lstlisting}[language=c]
    void eraseAll() {baseMap::erase(baseMap::begin(), baseMap::end()); }
\end{lstlisting}
Then give this command again:
\begin{lstlisting}[language=bash]
  $ ./install
\end{lstlisting}

This time,ns2 will be installed without any error.

Now give this command in the terminal
\begin{lstlisting}[language=bash]
  $ gedit ~/.bashrc 
\end{lstlisting}
After openning bashrc file,add this line in the file after the third line:(Remember,my ns-allinone-2.35 folder is in home directory.So all path should be changed according to the destination of your folder.)
\begin{lstlisting}[language=bash]
#LD_LIBRARY_PATH
OTCL_LIB=~/ns-allinone-2.35/otcl-1.14
NS2_LIB=~/ns-allinone-2.35/lib
X11_LIB=/usr/X11R6/lib
USR_LOCAL_LIB=/usr/local/lib
export LD_LIBRARY_PATH=$LD_LIBRARY_PATH:$OTCL_LIB:$NS2_LIB:$X11_LIB:$USR_LOCAL_LIB

 # TCL_LIBRARY
 TCL_LIB=~/ns-allinone-2.35/tcl8.5.10/library
 USR_LIB=/usr/lib
 export TCL_LIBRARY=$TCL_LIB:$USR_LIB

 # PATH
 XGRAPH=~/ns-allinone-2.35/bin:~/ns-allinone-2.35/tcl8.5.10/unix:~//ns-allinone-2.35/tk8.5.10/unix
 NS=~/ns-allinone-2.35/ns-2.35/
 NAM=~/ns-allinone-2.35/nam-1.15/
 PATH=$PATH:$XGRAPH:$NS:$NAM
\end{lstlisting}
Save the file and close it.

Give the following command in terminal:
\begin{lstlisting}[language=bash]
  $ source ~/.bashrc 
  $ ns
\end{lstlisting}
If you get \% sign,ns2 is successfully installed.

You can check the validity,by typing the following command:(assuming that the terminal is opend in ns-allinone-2.35/ns-2.35)
\begin{lstlisting}[language=bash]
  $ ./validate
\end{lstlisting}
It may take about 30 minutes and find that some tests fail.In that case,try this command:
\begin{lstlisting}[language=bash]
  $ sudo apt-get install libx11-dev xorg-dev libxmu-dev libperl4-corelibs-perl
\end{lstlisting}
It will install missing packege.
\chapter{Analysing Sample File}
We are going to analysis to sample file : 1. \textbf{802\_11\_udp.tcl} and 2. \textbf{802\_11.sh}
You can sample files with analysis here : \href{https://github.com/lsiddiqsunny/CSE-322-Computer-Networks-Sessional/tree/master/NS2\%20Offline/Sample}{\textbf{Sample Files}}

To run the script file,we have to execute\textbf{ 802\_11.sh} .

You may receive this type of output:

\includegraphics[width=1\textwidth]{ss1.png}
\end{document}